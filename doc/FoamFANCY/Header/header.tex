\documentclass[
  DIV=11,                 %% Satzspiegelkonstruktion, siehe KOMA-Script Anleitung
  fontsize=11pt,          %% Text in Schriftgrad 11pt
  BCOR=10mm,              %% Bindekorrektur, Klemmbindung verbraucht ca. 10mm
  captions=tableheading,  %% Tabellen haben eine Überschrift!
  headinclude=true,       %% Kopfzeile wird zum Satzspiegel gerechnet
  footinclude=false,      %% Fußzeile zählt zum Seitenrand (weil nur Seitenzahl in Fußzeile)
  headings=normal,        %% Überschriften etwas verkleinern (auf LARGE)
  numbers=noenddot,       %% kein Endpunkt bei Gliederungsnummerierung
  listof=flat,            %% Verzeichnisse der Gleitumgebungen nicht einrücken
  %draft,                  %% Draftmodus, keine Bilder/Farben
]{scrreprt}                


\usepackage[T1]{fontenc}      %% europäische Zeichensätze
\usepackage[utf8]{inputenc}   %% UTF-8 Kodierung, direkte Eingabe von Umlauten 
\usepackage[ngerman]{babel}   %% deutsche Anpassungen

%% Mathe-Pakete
%% ------------------------------------------------------------------------------
\usepackage{amsmath}       %% Matheumgebung
%\usepackage{amsfonts}     %% zusätzliche Mathe Schriften
%\usepackage{amssymb}      %% zusätzliche Symbole


%% noch mehr Symbole bei Bedarf
%% ------------------------------------------------------------------------------
\usepackage{latexsym}      %% zusätzliche Symbole
\usepackage{textcomp}      %% zusätzliche Symbole
\usepackage{eurosym}       %% Euro Symbol


\usepackage{siunitx}       %% richtige Schreibweise für SI Einheiten

\usepackage{setspace}      %% Zeilenabstand anpassen


%%------------------------------------------------------------------------------
%% Schriftauswahl
%%------------------------------------------------------------------------------
%\usepackage{lmodern}                 %% Latin Modern

%\usepackage{mathpazo}                %% Palatino

\usepackage[charter]{mathdesign}      %% Charter

%% für Charter und Palatino Zeilenabstand leicht erhöhen (1.05 bis 1.08)
\setstretch{1.08}            %% Einstellung schriftabhängig, für Latin Modern deaktivieren!

%% Bera Sans als serifenlose Schrift
\usepackage{berasans}

%% Bera Mono als monospaced Schrift (für fette Keywords in Listings)
\usepackage[scaled]{beramono}
%%------------------------------------------------------------------------------


%% Satzspiegel nach Schriftauswahl neu berechnen
\KOMAoptions{DIV=current}

%% Überschriften für alle Gliederungsebenen in fett und der gleichen
%% Schriftart wie den Standardtext setzen
\setkomafont{disposition}{\normalfont\bfseries}


%% Anpassung von Kopf- und Fußzeilen
%%------------------------------------------------------------------------------
\usepackage{scrpage2}        %% KOMA-Script Erweiterung für Kopf-/Fußzeilen
\pagestyle{scrheadings}      %% Seitenstil scrheadings verwenden
\clearscrheadfoot            %% alle Felder in Kopf- und Fußzeile löschen
\automark[chapter]{chapter}  %% Kopfzeile: Kapitelname außen


%% optischer Randausgleich
\usepackage{microtype}

%% korrektes Leerzeichen nach Makro
\usepackage{xspace}

%% einheitliche Anführungszeichen festlegen
%% Aufruf: \enquote{was in Anführungszeichen stehen soll}
\usepackage[
  babel,
  %german=guillemets,  %% spitze Klammern
  german=quotes,       %% normale Anführungszeichen
]{csquotes}


%% erweiterte Farbmöglichkeiten mit xcolor
%% Option [table] lädt das Package colortbl nach
\usepackage[table]{xcolor}

%% Abbildungen einbinden
\usepackage{graphicx}
\usepackage{wrapfig}

%% Zeilenabstand in Auflistungen verringern
%% Aufruf mit {enumerate*} bzw. {itemize*} 
\usepackage{mdwlist}


%% Tabellen
%% ------------------------------------------------------------------------------
\usepackage{booktabs}    %% für bessere Linien 	top-/mid-/bottomrule	
\usepackage{tabularx}    %% erweiterte Tabellendefinitionen 
\usepackage{longtable}  %% Lange Tabellen über mehr als eine Seite
\usepackage[bottom]{footmisc}  %% verhindert, dass Bilder unter Fußnoten rutschen


%% Tabellen- und Bildbeschreibungen anpassen
%% ------------------------------------------------------------------------------
\setlength\abovecaptionskip{2pt}    %% Abstand der Beschreibung zum Objekt
%% Schrift für Label und Beschreibungstext wie für normalen Text
\setkomafont{captionlabel}{\small\normalfont\bfseries}
\setkomafont{caption}{\small\normalfont}


%% Verarbeitung bibliografischer Daten mit Biblatex
%% Über Option  backend  entweder biber oder bibtex verwenden
%% Im Editor für Bib(la)tex folgenden Befehl eintragen:
%%     biber %    oder    bibtex % 
%% ------------------------------------------------------------------------------
\usepackage[
  style=numeric,   %% numerischer Stil
  isbn=false,      %% keine ISBN ausgeben
  url=false,       %% keine URL ausgeben
  backend=biber,   %% entweder biber zur Verarbeitung von Literaturverweisen 
  %backend=bibtex,   %% oder bibtex (veraltet, kann keine Umlaute verarbeiten) 
  %backref,         %% Rückverweise auf Zitate
]{biblatex}


%% Darstellung im Literaturverzeichnis mit biblatex auf Name, Vorname umstellen
%%------------------------------------------------------------------------------
\DeclareNameFormat{default}{%
\usebibmacro{name:last-first}{#1}{#3}{#5}{#7}%
\usebibmacro{name:andothers}}


%% Quellcode anzeigen, Standardeinstellungen
%% ------------------------------------------------------------------------------
\usepackage{listings}
\renewcommand{\lstlistlistingname}{Quelltextverzeichnis}    %% Umbenennen
\lstset{
  float=tb,
  captionpos=t,
%   basicstyle=\small\ttfamily,
  basicstyle=\tiny\ttfamily,
  keywordstyle=\bfseries,
  commentstyle=\ttfamily,
  columns=fixed,
  tabsize=2,
  %frame=single,      %% einfache Linie als Umrandung
  framerule=0.1pt,
  extendedchars=true,
  showspaces=false,
  showstringspaces=false,
  numbers=left,
  numberstyle=\sffamily\tiny,
  breaklines=true,
%  backgroundcolor=\color{black!10},      %% Hintergrundfarbe helles Grau
  breakautoindent=true,
  belowskip=2ex,
  literate=%         %% Umlaute wg. utf8 ersetzen z.B.  ü durch \"u
    {Ö}{{\"O}}1 {Ä}{{\"A}}1 {Ü}{{\"U}}1
    {ß}{{\ss}}1 {ü}{{\"u}}1 {ä}{{\"a}}1 {ö}{{\"o}}1
}


%% Listings-Umgebung für Fortan77 definieren, nur Inline Code!
%% Aufruf mit:
%% \begin{fortran}{Beschriftungstext}{lst:labelname}
%%   hier der FORTAN-CODE
%% \end{fortran}
%%--------------------------------------------------------------
\lstnewenvironment{fortran}[2]{%
  \lstset{caption=#1,label=#2,language={[77]Fortran}}
}
{}

\lstnewenvironment{dict}[2]{%
  \lstset{caption=#1,label=#2,language={C++}}
}
{}
%%--------------------------------------------------------------


%% Warnungen von KOMA-Script für Entwickler umgehen
\usepackage{scrhack}

%% Blindtext
\usepackage{blindtext}


%% Verweise/Links für PDF, möglichst zuletzt laden!
%-----------------------------------------------------------------------------
\usepackage[
  pdfborder={0 0 0},
  %colorlinks=true,    %% farbige Links am Bildschirm, false oder deaktiviert für Ausdruck
  pdfstartview=FitV,   %% wie soll Viewer starten?
  linkcolor=blue,      %% Farbe für Querverweise
  citecolor=blue,      %% Farbe für Zitierungen
  urlcolor=blue,       %% Farbe für Links
  bookmarks=true,      %% Lesezeichen im PDF
  bookmarksnumbered,   %% nummerierte Kapitel
  plainpages=false,
]{hyperref}


%% das offizelle Euro Symbol verwenden, muss nach hyperref stehen!
\let\texteuro\euro
