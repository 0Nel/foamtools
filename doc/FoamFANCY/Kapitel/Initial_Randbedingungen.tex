\chapter{Inital und Randbedingungen}

In diesem Bereich passieren die meisten Fehler; aus diesem Grund sollte hier mit besonderer Vorsicht und Konzentration gearbeitet werden.

Die Initial und Randbedingungen verteilen sich auf zwei Ordner: \textit{0} und \textit{constant}

Es müssen an allen Rändern Randbedingungen definiert werden.

Grundsätzlich unterscheidet \textsc{OpenFOAM} zwischen verschiedenen Patch-typen: \textit{base type}, \textit{primitve type} und \textit{derived type}. Dabei bedeutet \textit{base type} einfachste geometrische Definitionen. Es gibt nut zwei verschiedene Arten: \texttt{patch} und \texttt{wall}.

Aus dem \textit{base type} ergeben sich die \textit{primitive types} für die Größe $ \Phi $ (siehe \autoref{tab:primitive}): 
	
	\begin{table}[htb]
	  \centering
	  \begin{tabular}{m{2.5cm}m{6cm}m{3cm}} 
	  \toprule
	    	Typ & Beschreibung & Festzulegende Werte \\
	  \midrule
			\texttt{fixedValue} & Wert für $ \Phi $ muss festgelegt werden & \texttt{value} \\
			\texttt{fixedGradient} & Der Gradient für $ \Phi $ muss festgelegt werden & \texttt{gradient} \\
			\texttt{zeroGradient} & Der Gradient von $ \Phi $ ist null & - \\
			\texttt{calculated} & Die Werte für $ \Phi $ werden durch andere Größen bestimmt & - \\
			\texttt{mixed} & Mischt \texttt{fixedValue} und \texttt{fixedGradient} abhängig vom Wert in 	\texttt{valueFraction} & \texttt{refValue, refGradient, valueFraction, value} \\
			\texttt{directionMixed} & wie \texttt{valueFraction}, jedoch mit \texttt{valueFraction} als Tensor, sodass sich verschiedene Mischlevel in Normal- und Tangentialrichtung ergeben. & \texttt{refValue, refGradient, valueFraction, value} \\
	  \bottomrule
	  
	  \end{tabular}
	  \caption[Terme und ihre Bezeichnungen in OpenFOAM]{Terme und ihre Bezeichnungen in OpenFOAM}
	  \label{tab:primitive}
	\end{table} 

Aus der Kombination dieser Typen lassen sich die sog. \textit{derived types} herleiten. Da es insgesamt 67 dieser Randbedingungen gibt, werden nachfolgend nur die wichtigsten aufgeführt. 


SRFFreestreamVelocity
SRFVelocity
activeBaffleVelocity
activePressureForceBaffleVelocity
advective
atmBoundaryLayerInletVelocity
calculated
codedFixedValue
codedMixed
cyclic
cyclicACMI
cyclicAMI
cyclicSlip
cylindricalInletVelocity
directionMixed
empty
externalCoupled
fixedGradient
fixedInternalValue
fixedJump
fixedJumpAMI
fixedMean
fixedNormalInletOutletVelocity
fixedNormalSlip
fixedValue
flowRateInletVelocity
fluxCorrectedVelocity
freestream
inletOutlet
interstitialInletVelocity
kqRWallFunction
mapped
mappedField
mappedFixedInternalValue
mappedFixedPushedInternalValue
mappedFlowRate
mappedVelocityFlux
mixed
movingWallVelocity
nonuniformTransformCyclic
oscillatingFixedValue
outletInlet
outletMappedUniformInlet
outletPhaseMeanVelocity
partialSlip
pressureDirectedInletOutletVelocity
pressureDirectedInletVelocity
pressureInletOutletParSlipVelocity
pressureInletOutletVelocity
pressureInletUniformVelocity
pressureInletVelocity
pressureNormalInletOutletVelocity
processor
processorCyclic
rotatingPressureInletOutletVelocity
rotatingWallVelocity
sliced
slip
supersonicFreestream
surfaceNormalFixedValue
swirlFlowRateInletVelocity
symmetry
symmetryPlane
timeVaryingMappedFixedValue
translatingWallVelocity
turbulentInlet
uniformFixedGradient
uniformFixedValue
uniformInletOutlet
uniformJump
uniformJumpAMI
variableHeightFlowRateInletVelocity
waveTransmissive
wedge
zeroGradient

\newpage