\chapter{Strömungsmechanik}

\begin{itemize}
	\item Impulsgleichung
	\item Energiegleichung
	\item Masseerhaltung
	\item Impulsgleichung für inkompressible, newtonsche Fluide (N-S Gleichung)
	\item Impulsgleichung für kompressible, newtonsche Fluide 
\end{itemize}

Wir gehen zunächst von einem inkompressiblen, newtonschen Medium für das die Kontinuumshypothese gilt aus. Dadurch wird die Impulsgleichung zur Navier-Stokes-Gleichung (\autoref{eq:ns}):

\begin{equation}
	\label{eq:ns}
	\frac{\partial u_{i}}{\partial t} + u_{j} \frac{u_{i}}{x_{j}} = - \frac{ \frac{p}{\rho} + g k}{\partial x_{i}} + \nu \frac{\partial^{2} u_{i}}{\partial u_{j}^{2}}
\end{equation}

Durch die Inkompressibilität vereinfachst sich die Kontiuumsgleichung zu \autoref{eq:konti}:

\begin{equation}
	\label{eq:konti}
	\frac{\partial u_{i}}{\partial x_{i}} = 0
\end{equation}

Der erste Term von \autoref{eq:ns}, $ \frac{\partial u_{i}}{\partial t} $ wird als instationärer Term bezeichnet, er repräsentiert den Einfluss der Änderung des Impulses über die Zeit. Der zweite Term $ u_{j} \frac{u_{i}}{x_{j}} $, auch als konvektiver Term bezeichnet, beschreibt die Impulstransport aufgrund von Geschwindigkeitsänderungen. Dieser Term ist nichtlinear. 
Der dritte Term $ - \frac{ \frac{p}{\rho} + g k}{\partial x_{i}} $ ist der Druckgradient. In OpenFOAM wird der Druckgradient mit weiteren äußeren Kräften, wie z.B. der Gravitation (bei mehrphasigen Systemen) oder elektromagnetischen Kräften zusammengefasst. 
Der letzte Term $ \nu \frac{\partial^{2} u_{i}}{\partial u_{j}^{2}} $, auch als diffusiver Term bezeichnet. 

\newpage